\documentclass[a4paper,11pt]{report}
\usepackage[T1]{fontenc}
\usepackage[utf8]{inputenc}
\usepackage{lmodern}
\usepackage[francais]{babel}
\usepackage[usenames,dvipsnames,svgnames,table]{xcolor}
\usepackage[colorlinks,linkcolor={blue!30!black},citecolor={blue!50!black},urlcolor={blue!80!black}]{hyperref}
\usepackage{amsmath,array,graphicx,caption,lmodern,subcaption,tikz,url,xspace,wrapfig}
\usepackage{textcomp,rotating,epic,eepic,pdfpages}
\usepackage[top=2cm,left=2.5cm,right=2.5cm,bottom=2cm]{geometry} % Géométrie de la page, modifier selon le besoin
\usepackage[babel=true,kerning=true]{microtype}

% \pdfsuppresswarningpagegroup=1
\title{Rapport de Stage d'Application}


\begin{document}
\pagenumbering{gobble}  % Pas de numérotation
\begin{titlepage}
    \vspace*{50px}
    \includegraphics[height=80px]{Images/logo_phelma.pdf}
    \vspace*{-80px}
\begin{flushright}
%     \vspace*{60px}
    \includegraphics[height=65px]{Images/CIME.jpg}
\end{flushright}

\vspace*{2cm}

\begin{center}
\rule{\linewidth}{0.5mm}\\[0.4cm]
{\huge{\bfseries Compte Rendu}\\[0.4cm]
\textsc{TP Simulation électronique}\\[0.4cm]}
\rule{\linewidth}{0.5mm}\\[0.5cm]

\LARGE{\textsc{Nicolas Paillet, Félix Piédallu \& Giulia Rizzo}}\\[0.7cm]
\large{\textsc{2015-2016}}\\[2cm]

\Large{~}\\[1cm]
% \includegraphics[width=0.4\textwidth]{Images/CIME.jpg}\\[1cm]
%
% \large{}\\[2cm]
%

\end{center}
\end{titlepage}

\tableofcontents        % Table des matières avec liens, générée automatiquement.
\newpage
\pagenumbering{arabic}  % Numérotation de retour !


\chapter{Introduction}
Lors de la conception de composant à semiconducteurs, il est important d'avoir des outils de simulation pour avoir une idée des résultats et performances attendues. Il est également important de maîtriser ces outils de simulation pour réaliser ces simulations. Ce TP a pour but de nous initier à certains de ces outils en nous proposant de réaliser la simulation de composants MOS tel que ceux étudiés de manière théorique en cours. Ainsi, nous pourrons mettre des graphes  numériques sur les équations calculées théoriquement.

\chapter{Simulation}

\section{Initialisation de la simulation}

Tout d'abord, on utilise l'éditeur \texttt{deckbuild} afin de mettre en place la séquence de commandes à réaliser dans le logiciel \texttt{Atlas}. En premier lieu il faut créer un maillage 2D pour pouvoir y inclure des régions.
\vspace{0.3cm}

\fbox{
\begin{minipage}{0.90\textwidth}
\noindent\texttt{go atlas\\mesh space.mult = 1.0\\x.mesh loc=0.0 spac=0.001\\x.mesh loc=0.1 spac=0.001\\x.mesh loc=0.2 spac=0.001\\x.mesh loc=0.3 spac=0.001\\y.mesh loc=0.000 spac=0.0001\\y.mesh loc=0.002 spac=0.0001\\y.mesh loc=1.002 spac=0.01}
\end{minipage}}
\vspace{0.3cm}

Il faut ensuite dessiner le composant que nous voulons modéliser sur cette grille. Pour cela, il nous suffira de dessiner des régions en précisant les matériaux utilisés pour chacune d'entre elles.
\vspace{0.3cm}

\fbox{
\begin{minipage}{0.95\textwidth}
\noindent\texttt{region number = 1 x.min=0.0 x.max = 0.3 y.min = 0.0 y.max = 0.002 material = Oxide\\
region number = 2 x.min=0 x.max = 0.3 y.min = 0.002 y.max = 1.002 material = Silicon}
\vspace{0.3cm}

\noindent\texttt{electrode name = gate   number = 1 x.min = 0.1 x.max = 0.2 y.min = 0.00 y.max = 0\\
electrode name = source number = 2 x.min = 0.0 x.max = 0.0 y.min = 0.002 y.max = 0.012\\
electrode name = drain  number = 3 x.min = 0.3 x.max = 0.3 y.min = 0.002 y.max = 0.012}
\vspace{0.3cm}

\noindent\texttt{doping uniform conc = 1E15 p.type region = 2\\
doping uniform conc = 1E20 N.type region = 2 x.left = 0.0 x.right = 0.1 y.max = 0.012\\
doping uniform conc = 1E20 N.type region = 2 x.left = 0.2 x.right = 0.3 y.max = 0.012\\
struct outf = mos.str}
\end{minipage}}
\vspace{0.3cm}

%\begin{tikzpicture}

%Dessiner ici le transistor qu'on modélise

%\end{tikzpicture}
\begin{figure}
\centering
\includegraphics[width=300pt]{Simu1-Dopage.png}
\caption{Image des différentes régions et dopages générés par le logiciel}
\label{transistortonyplot}
\end{figure}

Nous avons ainsi une modélisation 2D du transistor, comme le montre la Figure \ref{transistortonyplot}.

On précise ensuite au logiciel les différents contacts que nous allons utiliser, on obtient ainsi l'image en Figure \ref{TransistorFull}

\begin{figure}
    \centering
    \includegraphics[width=300pt]{TransistorFull.png}
    \caption{Image des régions, du dopage ainsi que des contacts générée par le logiciel}
    \label{TransistorFull}
\end{figure}

Maintenant que notre transistor est modélisé, nous allons pouvoir réaliser des simulations en jouant avec les différentes tensions appliquées.

\section{Tracés de caractéristiques}

En effet, il suffit ensuite d'ajouter quelques lignes au script afin de tracer des courbes.



\section{Optimisation du maillage}

Nous avons pu voir lors des calculs précédents qu'avec le maillage proposé initialement, l'obtention de points est très longue. En effet, le maillage est beaucoup trop fin, en particulier à des endroits où il n'a pas besoin de l'être. Ainsi, nous devons trouver un meilleur maillage afin d'optimiser le temps de calcul sans perdre de précision dans les zones importantes. 








\end{document}
