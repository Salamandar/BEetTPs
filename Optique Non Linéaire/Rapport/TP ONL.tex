\documentclass[a4paper,11pt]{report}
\usepackage[T1]{fontenc}
\usepackage[utf8]{inputenc}
\usepackage[francais]{babel}
\usepackage[babel=true,kerning=true]{microtype}
\usepackage[usenames,dvipsnames,svgnames,table]{xcolor}
\usepackage[colorlinks,linkcolor={blue!30!black},citecolor={blue!50!black},urlcolor={blue!80!black}]{hyperref}
\usepackage{amsmath,amsfonts,amssymb,array,graphicx,caption,lmodern,subcaption,tikz,url,xspace,wrapfig}
\usepackage{textcomp,rotating,epic,pdfpages,listings,diagbox,multirow,float}
\usepackage{pgfplots}
\pgfplotsset{width=7cm,compat=1.8}

\usepackage[top=25mm,bottom=25mm,left=25mm,right=25mm]{geometry}
\parskip=6pt % Espacement vertical entre les paragraphes

\begin{document}
\pagenumbering{gobble}  % Pas de numérotation
\begin{titlepage}
    \vspace*{50px}
    \includegraphics[height=80px]{Images/logo_phelma.pdf}
    \vspace*{-80px}
\begin{flushright}
%     \vspace*{60px}
    \includegraphics[height=65px]{Images/CIME.jpg}
\end{flushright}

\vspace*{2cm}

\begin{center}
\rule{\linewidth}{0.5mm}\\[0.4cm]
{\huge{\bfseries Compte Rendu}\\[0.4cm]
\textsc{TP Simulation électronique}\\[0.4cm]}
\rule{\linewidth}{0.5mm}\\[0.5cm]

\LARGE{\textsc{Nicolas Paillet, Félix Piédallu \& Giulia Rizzo}}\\[0.7cm]
\large{\textsc{2015-2016}}\\[2cm]

\Large{~}\\[1cm]
% \includegraphics[width=0.4\textwidth]{Images/CIME.jpg}\\[1cm]
%
% \large{}\\[2cm]
%

\end{center}
\end{titlepage}

\tableofcontents        % Table des matières avec liens, générée automatiquement.
\newpage
\pagenumbering{arabic}  % Numérotation de retour !

\chapter*{Introduction} \addcontentsline{toc}{chapter}{Introduction}
Nous avons vu lors de notre cours d'Optique Non Linéaire que certains cristaux ne se comportent pas de manière isotrope et linéaire lorsqu'on leur impose un rayonnement Laser.

Le but de ce TP est donc d'étudier le comportement d'un cristal de RbTiOPO$_4$ (RTP).

Dans une première partie nous avons utilisé un laser He:Ne de longueur d'onde $\lambda = 632.8nm$ pour faire apparaître le caractère anisotrope du cristal qui se traduit par la séparation des deux composantes de polarisation de la lumière par le cristal (Chapitres \ref{OCLTheorie} et \ref{OCLExp}).

La seconde partie portera sur le calculs des conditions de Génération de seconde harmonique (SHG) (Chapitre \ref{ONLTheorie}) puis l'utilisation d'un laser NdYAG ($\lambda = 1064nm$, invisible) qui nous a permis d'observer le caractère non linéaire du cristal par un doublage de fréquence de l'onde incidente de l'infrarouge vers la lumière visible (Chapitre \ref{ONLExp}).

\chapter{Optique Cristalline Linéaire (Théorie)} \label{OCLTheorie}
\section{Surface des indices}
Le cristal de RbTiOPO$_4$ (RTP) cristallise dans un système orthorhombique, et admet une symétrie d'orientation $mm2$.

Ce n'est donc pas un cristal centro-symétrique: il est anisotrope, et de classe optique biaxe. Cela veut dire que la susceptibilité électrique linéaire $\chi^{(1)}$ n'est pas un scalaire comme dans un milieu isotrope, mais un tenseur de rang 2 (une matrice), dépendant de la direction de propagation de l'onde incidente. Cette matrice peut être diagonalisée dans le repère diélectrique:
\[\chi^{(1)} = 
\left( \begin{matrix}
\chi^{(1)}_{xx}   & 0                 & 0 \\
0                 & \chi^{(1)}_{yy}   & 0 \\
0                 & 0                 & \chi^{(1)}_{zz}
\end{matrix}\right) \]

Un milieu isotrope aura ses trois valeurs propres égales, un milieu biaxe aura $\chi^{(1)}_{xx} = \chi^{(1)}_{yy} \neq \chi^{(1)}_{zz}$, et un milieu biaxe aura ses trois valeurs propres différentes.

L'indice de réfraction du milieu est directement lié à la susceptibilité électrique: dans un milieu isotrope, on a $n = \sqrt{1+\chi^{(1)}}$.

Dans un milieu anisotrope, cela devient un peu plus complexe. On va commencer par définir les indices de réfraction principaux :
\[  n_i = \sqrt{+\chi^{(1)}_{ii}} \text{, avec } i=\{x, y, z\} \]

On peut alors écrire l'équation de Fresnel, tirée des équations de Maxwell :
\[
    \dfrac{\sin^2(\theta)\cdot\cos^2(\phi)}{\dfrac{1}{n^2(\theta, \phi, \lambda)} - \dfrac{1}{n^2_x(\lambda)}}
 +  \dfrac{\sin^2(\theta)\cdot\sin^2(\phi)}{\dfrac{1}{n^2(\theta, \phi, \lambda)} - \dfrac{1}{n^2_y(\lambda)}}
 +  \dfrac{\cos^2(\theta)                 }{\dfrac{1}{n^2(\theta, \phi, \lambda)} - \dfrac{1}{n^2_z(\lambda)}}
    = 0
\]

Cette équation quadratique en $n^2$ admet deux solutions :
\[
    n^+(\theta, \phi, \lambda) = \dfrac{\sqrt 2}{\sqrt{-B-B\sqrt{B^2-4C}}}
    \text{ et }
    n^-(\theta, \phi, \lambda) = \dfrac{\sqrt 2}{\sqrt{-B+B\sqrt{B^2-4C}}}
\]

avec : 
\[ B =  -sin^2(\theta) cos^2(\phi)  \left[\dfrac{1}{n^2_y(\lambda)} + \dfrac{1}{n^2_z(\lambda)}\right]
        -sin^2(\theta) sin^2(\phi)  \left[\dfrac{1}{n^2_x(\lambda)} + \dfrac{1}{n^2_z(\lambda)}\right]
        -cos^2(\theta)              \left[\dfrac{1}{n^2_x(\lambda)} + \dfrac{1}{n^2_y(\lambda)}\right]
\]

et :
\[ C =  \dfrac{sin^2(\theta) cos^2(\phi)}{n^2_y(\lambda)n^2_z(\lambda)}
     +  \dfrac{sin^2(\theta) sin^2(\phi)}{n^2_x(\lambda)n^2_z(\lambda)}
     +  \dfrac{cos^2(\theta)            }{n^2_x(\lambda)n^2_y(\lambda)}
\]

Cela définit alors deux nappes intérieure et extérieure, car $n^+ \geq n^-$ est toujours vérifié.

Dans le cas d'un milieu biaxe, les surfaces d'indice ont cette forme :

TODO inclure image générée je ne sais comment en 3D.

\section{Aparallélisme des champs}
\section{Séparation des polarisations avec un cristal parallélépipédique}
\section{Séparation des polarisations avec un cristal cylindrique}

\chapter{Optique Cristalline Linéaire (Expérience)} \label{OCLExp}
\section{Mise en place de l'expérience}
\section{Orientation du repère optique}
\section{Mesure de la distance entre les spots de polarisations}

% Idem

\chapter{Optique Non-Linéaire: Conversion de fréquences (Théorie)}\label{ONLTheorie}
\section{Calculs de $\chi$ (Type I et Type II)}
\section{Calcul des angles pour le Type II}


\chapter{Optique Non-Linéaire: Conversion de fréquences (Expérience)} \label{ONLExp}
\section{Mise en place de l'expérience pour une SHG de type II}
\section{Mesure des angles d'accord de phase}

\chapter*{Conclusion} \addcontentsline{toc}{chapter}{Conclusion}
Au cours de ce TP nous avons pu mettre en évidence les concepts décrits théoriquement en cours. Nous avons pu observer grâce au laser et à sa forme cylindrique spécifique, l'anisotropie du cristal de RTP qui sépare les différentes polarisations de la lumière, mais également les effets non-linéaires qui peuvent intervenir sous certaines conditions. Nous avons réalisé ces expériences en accord avec la théorie, en vérifiant celle-ci grâce aux expérimentations.


\end{document}
