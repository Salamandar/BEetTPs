\documentclass[a4paper,11pt]{report}
\usepackage[T1]{fontenc}
\usepackage[utf8]{inputenc}
\usepackage[francais]{babel}
\usepackage[babel=true,kerning=true]{microtype}
\usepackage[usenames,dvipsnames,svgnames,table]{xcolor}
\usepackage[colorlinks,linkcolor={blue!30!black},citecolor={blue!50!black},urlcolor={blue!80!black}]{hyperref}
\usepackage{amsmath,amsfonts,amssymb,array,graphicx,caption,lmodern,subcaption,tikz,url,xspace,wrapfig}
\usepackage{textcomp,rotating,epic,pdfpages,listings,diagbox,multirow,float}
\usepackage{pgfplots}
\pgfplotsset{width=7cm,compat=1.8}

\usepackage[top=25mm,bottom=25mm,left=25mm,right=25mm]{geometry}
\parskip=6pt % Espacement vertical entre les paragraphes

\begin{document}
\pagenumbering{gobble}  % Pas de numérotation
\begin{titlepage}
    \vspace*{50px}
    \includegraphics[height=80px]{Images/logo_phelma.pdf}
    \vspace*{-80px}
\begin{flushright}
%     \vspace*{60px}
    \includegraphics[height=65px]{Images/CIME.jpg}
\end{flushright}

\vspace*{2cm}

\begin{center}
\rule{\linewidth}{0.5mm}\\[0.4cm]
{\huge{\bfseries Compte Rendu}\\[0.4cm]
\textsc{TP Simulation électronique}\\[0.4cm]}
\rule{\linewidth}{0.5mm}\\[0.5cm]

\LARGE{\textsc{Nicolas Paillet, Félix Piédallu \& Giulia Rizzo}}\\[0.7cm]
\large{\textsc{2015-2016}}\\[2cm]

\Large{~}\\[1cm]
% \includegraphics[width=0.4\textwidth]{Images/CIME.jpg}\\[1cm]
%
% \large{}\\[2cm]
%

\end{center}
\end{titlepage}

\tableofcontents        % Table des matières avec liens, générée automatiquement.
\newpage
\pagenumbering{arabic}  % Numérotation de retour !

\chapter*{Introduction} \addcontentsline{toc}{chapter}{Introduction}
Nous avons vu lors de notre cours d'Optique Non Linéaire que certains cristaux ne se comportent pas de manière isotrope et linéaire lorsqu'on leur impose un rayonnement Laser.

Le but de ce TP est donc d'étudier le comportement d'un cristal de RbTiOPO$_4$ (RTP).

Dans une première partie nous avons utilisé un laser He:Ne de longueur d'onde $\lambda = 632.8nm$ pour faire apparaître le caractère anisotrope du cristal qui se traduit par la séparation des deux composantes de polarisation de la lumière par le cristal (Chapitres \ref{OCLTheorie} et \ref{OCLExp}).

La seconde partie portera sur le calculs des conditions de Génération de seconde harmonique (SHG) (Chapitre \ref{ONLTheorie}) puis l'utilisation d'un laser NdYAG ($\lambda = 1064nm$, invisible) qui nous a permis d'observer le caractère non linéaire du cristal par un doublage de fréquence de l'onde incidente de l'infrarouge vers la lumière visible (Chapitre \ref{ONLExp}).

\chapter{Optique Cristalline Linéaire (Théorie)} \label{OCLTheorie}
\section{Surface des indices}
\subsection{Introduction: la susceptibilité électrique}
Dans un milieu quelconque, la susceptibilité électrique linéaire $\chi^{(1)}$ n'est pas forcément un scalaire. On peut de manière générale le représenter par un tenseur de rang 2 (i.e une matrice), dépendant de la direction de propagation de l'onde incidente.

Ce tenseur est diagonalisable dans le repère diélectrique, qui peut être totalement différent du repère cristallographique. On l'écrit alors sous cette forme:
\[\chi^{(1)} = 
\left( \begin{matrix}
\chi^{(1)}_{xx}   & 0                 & 0 \\
0                 & \chi^{(1)}_{yy}   & 0 \\
0                 & 0                 & \chi^{(1)}_{zz}
\end{matrix}\right) \]

La complexité de cette matrice est dépendante des symétries du cristal considéré: Plus il est de haute symétrie, plus on aura de valeurs propres égales:\\
Un milieu isotrope aura ses trois valeurs propres égales, un milieu biaxe aura $\chi^{(1)}_{xx} = \chi^{(1)}_{yy} \neq \chi^{(1)}_{zz}$, et un milieu biaxe aura ses trois valeurs propres différentes.

Le cristal de RbTiOPO$_4$ (RTP) cristallise dans un système orthorhombique, et admet une symétrie d'orientation $mm2$. Il est peu symétrique, il est alors de classe optique biaxe.

De plus il est orthorhombique, le repère cristallographique sera alors parallèle au repère diélectrique.
\vspace*{5mm}

\subsection{L'indice de réfraction}
L'indice de réfraction du milieu est directement lié à la susceptibilité électrique: dans un milieu isotrope, on a $n = \sqrt{1+\chi^{(1)}}$.

Dans un milieu anisotrope, cela devient un peu plus complexe. On va commencer par définir les indices de réfraction principaux:
\[  n_i = \sqrt{1+\chi^{(1)}_{ii}} \text{, avec } i=\{x, y, z\} \]

On peut alors écrire l'équation de Fresnel, tirée des équations de Maxwell:
\[
    \dfrac{\sin^2(\theta)\cdot\cos^2(\phi)}{\dfrac{1}{n^2(\theta, \phi, \lambda)} - \dfrac{1}{n^2_x(\lambda)}}
 +  \dfrac{\sin^2(\theta)\cdot\sin^2(\phi)}{\dfrac{1}{n^2(\theta, \phi, \lambda)} - \dfrac{1}{n^2_y(\lambda)}}
 +  \dfrac{\cos^2(\theta)                 }{\dfrac{1}{n^2(\theta, \phi, \lambda)} - \dfrac{1}{n^2_z(\lambda)}}
    = 0
\]

Cette équation quadratique en $n^2$ admet deux solutions:
\[
    n^+(\theta, \phi, \lambda) = \dfrac{\sqrt 2}{\sqrt{-B-B\sqrt{B^2-4C}}}
    \text{ et }
    n^-(\theta, \phi, \lambda) = \dfrac{\sqrt 2}{\sqrt{-B+B\sqrt{B^2-4C}}}
\]

avec: 
\[ B =  -sin^2(\theta) cos^2(\phi)  \left[\dfrac{1}{n^2_y(\lambda)} + \dfrac{1}{n^2_z(\lambda)}\right]
        -sin^2(\theta) sin^2(\phi)  \left[\dfrac{1}{n^2_x(\lambda)} + \dfrac{1}{n^2_z(\lambda)}\right]
        -cos^2(\theta)              \left[\dfrac{1}{n^2_x(\lambda)} + \dfrac{1}{n^2_y(\lambda)}\right]
\]

et:
\[ C =  \dfrac{sin^2(\theta) cos^2(\phi)}{n^2_y(\lambda)n^2_z(\lambda)}
     +  \dfrac{sin^2(\theta) sin^2(\phi)}{n^2_x(\lambda)n^2_z(\lambda)}
     +  \dfrac{cos^2(\theta)            }{n^2_x(\lambda)n^2_y(\lambda)}
\]

\subsection{Surface des indices}
Cela définit alors deux nappes intérieure et extérieure, car $n^+ \geq n^-$ est toujours vérifié.

Dans le cas d'un milieu biaxe, les surfaces d'indice ont cette forme :

TODO inclure image générée je ne sais comment en 3D.

\paragraph{L'équation de ces nappes} sur le plan $(\vec x, \vec z)$ est de la forme:
\[  \frac{1}{n^2_\text{ellipse}} = \frac{\cos^2(\theta)}{n^2_x} + \frac{\sin^2(\theta)}{n^2_z} \]
\[  n^2_\text{cercle} = n_y \]

La direction de propagation de l'onde électromagnétique est déterminée par l'intersection des nappes $n^+$ et $n^-$, c'est-à-dire par la solution double de l'équation de Fresnel:
\[  n(\theta, \phi, \lambda) = \dfrac{\sqrt 2}{\sqrt{-B}} \]

L'angle de propagation $V_z$ est alors défini par:
\[ \sin^2(V_z) = \frac{n_y^{-2}-n_x^{-2}}{n_z^{-2}-n_x^{-2}} \]


\section{Aparallélisme des champs}
Une susceptibilité électrique non scalaire implique alors que la polarisation du milieu ne soit pas parallèle au champ électrique:
\[\vec{P} = \epsilon_0 \chi^{(1)} \cdot\vec{E}\]

et: 
\[\vec{D} = \epsilon_0 \vec{E} + \vec{P}
          = \epsilon_0 (1+\chi^{(1)}) \cdot\vec{E}
          = \epsilon_0 \epsilon_r^{(1)} \vec{E}\]

Enfin, on a:
\begin{itemize}
    \item $\vec{\pi} \perp \vec{E}$
    \item $\vec{k}   \perp \vec{D}$
    \item $\vec{\pi} = \frac{1}{\mu_0} \vec{E}\wedge \vec{B}$
\end{itemize}
TODO schéma vecteurs

Il faut garder en tête que l'on observe physiquement $\vec{\pi}$ et $\vec{E}$, et PAS $\vec{k}$ ni $\vec{D}$.

\section{Séparation des polarisations avec un cristal parallélépipédique}
À l'interface entre un milieu isotrope (l'air) et un milieu anisotrope (notre cristal RTP), il y a une variation d'indice optique anisotrope:
\begin{itemize}
    \item $\vec{k}$ et $\vec{D}$ ne varient pas: l'onde conserve la même direction (TODO préciser)
    \item $\vec{\pi}$ et $\vec{E}$ dévient: l'onde change de direction de propagation 
\end{itemize}

Seule la direction de propagation change: $\vec{D}$ reste orienté selon la direction originelle de propagation.

TODO schéma cristal

En sortie du cristal, on a le même phénomène qu'en entrée: les deux faisceaux se retrouvent parallèles, mais séparés par une distance $\delta_z$.

Or l'angle $\rho$ est très faible habituellement, $\delta_z$ aussi. Cela ne nous permet pas d'effectuer une mesure précise.
\section{Séparation des polarisations avec un cristal cylindrique}
Une autre méthode est alors d'utiliser un cristal cylindrique.

TODO schéma cristal

Comme $\rho_{ext}$ est assez grand, la mesure peut être très précise ; elle peut être améliorée en éloignant le capteur du cristal.

C'est cette configuration que nous avons utilisé dans la suite du TP.

\chapter{Optique Cristalline Linéaire (Expérience)} \label{OCLExp}
\section{Mise en place de l'expérience}
\section{Orientation du repère optique}
\section{Mesure de la distance entre les spots de polarisations}

% Idem

\chapter{Optique Non-Linéaire: Conversion de fréquences (Théorie)}\label{ONLTheorie}

Maintenant que nous avons vu les effets de l'anisotropie du cristal en réponse linéaire, il nous faut étudier la non-linéarité, notamment la Génération de Seconde Harmonique (SHG). La Génération de Seconde Harmonique est un cas dégénéré de fusion de photons (en considérant l'apsect corpusculaire), elle correspond à l'intéraction de deux photons de fréquence $\omega$ qui donnent un photon de fréquence 2$\omega$.

\section{Accord de phase}

%je sais pas comment formuler avant ça

$\Delta k$ est le déphasage entre la polarisation non-linéaire et le champs qu'elle génère. La condition d'accord de phase revient à dire:

\[\Delta k=k_3-k_2-k_1=0\]

Or:

\[k_i=\dfrac{2\pi}{\lambda_i}n(\lambda_i)\]

On peut ainsi réécrire $\Delta k$:

\[\Delta k=2\pi\left[\dfrac{n^{\pm}(\lambda_3,\theta,\phi)}{\lambda_3}-\dfrac{n^{\pm}(\lambda_2,\theta,\phi)}{\lambda_2}-\dfrac{n^{\pm}(\lambda_1,\theta,\phi)}{\lambda_1}\right]\]

Il y a ainsi 8 possibilités en focntion du choix de + ou de - pour $n^{\pm}$:

\begin{center}
\begin{tabular}{|c|c|c|}
$\lambda_3$&$\lambda_2$&$\lambda_1$\\
\hline
+&+&+\\
+&+&-\\
+&-&-\\
\hline
-&+&+\\
-&+&-\\
-&-&+\\
\hline
-&-&-\\
\end{tabular}
\end{center}

Les 4 premiers cas sont impossibles car $\lambda_3<\lambda_{1,2}$, or d'après les courbes de $n^+$ et $n^-$, on ne peut avoir $n^+$ avec un longueur d'onde plus faible. Pour des raisons de bijectivité, avoir le même signe partout est impossible. Ce qui laisse 3 possibilités, dont deux dégénérées dans le cas $\lambda_1=\lambda_2$. On notera Type I le cas "-,+,+" et Type II (confondu avec Type III dans notre cas) le cas "-,-,+" (ou "-,+,-").

On remarque ainsi que pour le Type I, on injecte sur les nappes $n^+$, il faudra cond polariser horizontalement, alors que pour le Type II on injecte à la fois sur $n^+$ et $n^-$, il faudra donc un polariseur à 45\textdegree pour injecter simultanément sur ces deux nappes. La polarisation appliquée impose le désaccord de phase, d'où la nécessité d'être capable de contrôler la polarisation (laser polarisé et utilisation systématique de lame $\dfrac{\lambda}{2}$) lors des manipulations.

\section{Condition d'accord de phase dans le cas de l'expérience}

Il nous faut à présent traduire la condiction de phase sur des variables modifiables dans le cas de l'expérience. Nous travaillerons dans le plan YZ avec des longueurs d'ondes suivantes:

\indent$\cdot\,\lambda_1=\lambda_2=1,064\mu m=\lambda$\\
\indent$\cdot\,\lambda_3=0,532\mu m=\dfrac{\lambda}{2}$\\

\paragraph{Pour le Type I}
\begin{align*}
\Delta k&=2\pi\left[\dfrac{n^-(\lambda_3,\theta,\phi)}{\lambda_3}-\dfrac{n^+(\lambda_2,\theta,\phi)}{\lambda_2}-\dfrac{n^+(\lambda_1,\theta,\phi)}{\lambda_1}\right]\\
\Delta k&=\dfrac{2\pi\times 2}{\lambda}\left[n^-\left(\dfrac{\lambda}{2},\theta,\dfrac{\pi}{2}\right)-n^+\left(\lambda,\theta,\dfrac{\pi}{2}\right)\right]
\end{align*}

Or: 
\begin{align*}
    &\cdot n^-\left(\dfrac{\lambda}{2},\theta,\dfrac{\pi}{2}\right)=n_x\left(\dfrac{\lambda}{2}\right)\\
    &\cdot n^+\left(\lambda,\theta,\dfrac{\pi}{2}\right)=\dfrac{1}{\sqrt{\dfrac{\cos^2\theta}{n_y^2(\lambda)}+\dfrac{\sin^2\theta}{n_z^2(\lambda)}}}
\end{align*}

Ainsi, après calculs, on obtient: 
\[\sin^2\theta_{I}=\dfrac{\dfrac{1}{n_x^2(\lambda/2)}-\dfrac{1}{n_y^2(\lambda)}}{\dfrac{1}{n_z^2(\lambda)}-\dfrac{1}{n_y^2(\lambda)}}\]

D'où: 
\[\theta_{I}=41^{\circ}\]

\paragraph{Pour le Type II}
\[\Delta k=\dfrac{2\pi}{\lambda}\left[2n^-\left(\dfrac{\lambda}{2},\theta,\dfrac{\pi}{2}\right)-n^-\left(\lambda,\theta,\dfrac{\pi}{2}\right)-n^+\left(\lambda,\theta,\dfrac{\pi}{2}\right)\right]\]

Avec les mêmes expressions pour $n^-(\lambda/2)$ et $n^+(\lambda)$ mais avec:
\[\cdot\,n^-(\lambda)=n_x(\lambda)\]

Ainsi, après calculs, on obtient cette fois:
\[\sin^2\theta_{II}=\dfrac{\dfrac{1}{(2n_x(\lambda/2)-n_x(\lambda))^2}-\dfrac{1}{n_y^2(\lambda)}}{\dfrac{1}{n_z^2(\lambda)}-\dfrac{1}{n_y^2(\lambda)}}\]

D'où:
\[\theta_{II}=76^{\circ}\]

La condtion d'accord de phase se traduit donc par un angle $\theta$ imposé. Cependant, l'accord de phase traduit un maximum d'intéractions entre les 3 photons. Ainsi, les angles trouvées devraient être des maxima d'intensité de lumière à $2\omega$ et non pas des directions exlusives (même si c'est assez séléctif, comme nous le verrons).

\section{Calculs de $\chi$ (Type I et Type II)}

Nous avons à présent les angles $\theta$ dans lesquelles on devrait voir apparaitre la génération de seconde harmonique, donc une lumière de longueur d'onde égale à $0,532\mu m$, pour les Types I et II. Cependant, il reste un point important à vérifier: le  cristal possède-t-il un coefficient de susceptibilité electrique non linéaire non nul pour ces directions ? C'est ce que nous allons vérifier grâce au calcul de $\chi_{eff}$.

\paragraph{Pour le Type I}
On ecrit l'expression de la susceptibilité effective:

\[\chi_{eff}^{SHG_I}=\sum_{i,j,k}e^-_j(\lambda/2)\chi_{j,i,k}(\lambda/2)e^+_i(\lambda)e^+_k(\lambda)\]

Avec {i,j,k}={x,y,z} car $\chi^{(2)}$ est un tenseur de 27 élements, mais beaucoup sont nuls, pour des raisons de symétrie du cristal. On se réfère alors à des tables de $\chi^{(2)}$ qui nous fournissent les éléments non nuls, notamment pour le groupe de symétrie 2mm auquel le RTP appartient. Mais avant de regarder cette table, on réduit d'abord les possibilités avec le produit vectoriel des polarisations: $e^-_j(\lambda/2)e^+_i(\lambda)e^+_k(\lambda)$.

En effet, on note:
\[f_{j,i,k}=\begin{pmatrix}-1\\0\\0\end{pmatrix}\begin{pmatrix}0\\-\cos\theta'\\\sin\theta'\end{pmatrix}\begin{pmatrix}0\\-\cos\theta'\\\sin\theta'\end{pmatrix}\]

Les coefficients $f_{j,i,k}$ sont presque tous nuls sauf $f_{x,y,y}$, $f_{x,z,z}$, $f_{x,y,z}$ et $f_{x,z,y}$. 
On peut à présent comparer avec les coefficients non nuls de $\chi^{(2)}$. Or, d'après la table, les quatres coefficients correspondants sont nuls.

Cela implique:
\[\chi_{eff}^{SHG_I}=0\]

Ainsi, la symétrie du cristal de RTP empêche la génération de seconde harmonique en polarisant la lumière selon les nappes $n^+$. On ne verra donc aucune lumière à $0,532\mu m$ en polarisant horizontalement uniquement. Il reste à réaliser la même étude pour le Type II.

\paragraph{Pour le Type II}

On reprend la même étude avec l'expression: 

\[\chi_{eff}^{SHG_I}=\sum_{i,j,k}\chi_{j,i,k}(\lambda/2)e^-_j(\lambda/2)e^+_i(\lambda)e^-_k(\lambda)\]

Dans ce cas:

\[f_{j,i,k}=\begin{pmatrix}-1\\0\\0\end{pmatrix}\begin{pmatrix}0\\-\cos\theta'\\\sin\theta'\end{pmatrix}\begin{pmatrix}-1\\0\\0\end{pmatrix}\]

Alors, $f_{x,y,x}$ et $f_{x,z,x}$ sont non nuls, et d'après les tables $\chi_{x,y,x}$ est nul mais $\chi_{x,z,x}$ ne l'est pas. 

Comme $f_{x,z,x}=\sin\theta'$, on peut écrire:

\[\chi_{eff}^2=\chi_{x,z,x}^2\sin^2\theta'\neq 0\]

On peut donc avoir de la génération de seconde harmonique de Type II sur le cristal de RTP. Cependant, on remarque que $\chi_{eff}$ est maximal pour $\theta'\simeq\theta=90$\textdegree, or la condition d'accord de phase nous impose $\theta=76$\textdegree. Le rendement ne sera donc pas idéal, pour s'en approcher, il faudrait changer les longueurs d'ondes en jeu pour modifier l'angle imposé par la condition d'accord de phase (cf courbe d'accord de phase). Cependant, il est impossible d'avoir 0\% de rendement d'après ces mêmes courbes car $\theta=0$ n'est jamais solution de la condition d'accord de phase.

Il nous faut alors mettre en place l'expérience de SHG Type II afin de valider ces calculs, notamment vérifier l'angle trouvé.

\chapter{Optique Non-Linéaire: Conversion de fréquences (Expérience)} \label{ONLExp}
Après avoir établit certains paramètres pour l'expérience, il faut la mettre en place afin de les vérifier.
\section{Mise en place de l'expérience pour une SHG de type II}
Le montage ressemblera à celui vu précédemment, avec des miroirs pour calibrer le faisceau comme on le souhaite. 

%photo ?
\section{Mesure des angles d'accord de phase}

\chapter*{Conclusion} \addcontentsline{toc}{chapter}{Conclusion}
Au cours de ce TP nous avons pu mettre en évidence les concepts décrits théoriquement en cours. Nous avons pu observer grâce au laser et à sa forme cylindrique spécifique, l'anisotropie du cristal de RTP qui sépare les différentes polarisations de la lumière, mais également les effets non-linéaires qui peuvent intervenir sous certaines conditions. Nous avons réalisé ces expériences en accord avec la théorie, en vérifiant celle-ci grâce aux expérimentations.


\end{document}
