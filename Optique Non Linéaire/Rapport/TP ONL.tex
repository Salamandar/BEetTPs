\documentclass[a4paper,11pt]{report}
\usepackage[T1]{fontenc}
\usepackage[utf8]{inputenc}
\usepackage[francais]{babel}
\usepackage[babel=true,kerning=true]{microtype}
\usepackage[usenames,dvipsnames,svgnames,table]{xcolor}
\usepackage[colorlinks,linkcolor={blue!30!black},citecolor={blue!50!black},urlcolor={blue!80!black}]{hyperref}
\usepackage{amsmath,amsfonts,amssymb,array,graphicx,caption,lmodern,subcaption,tikz,url,xspace,wrapfig}
\usepackage{textcomp,rotating,epic,pdfpages,listings,diagbox,multirow,float}
\usepackage{pgfplots}
\pgfplotsset{width=7cm,compat=1.8}

\usepackage[top=25mm,bottom=25mm,left=25mm,right=25mm]{geometry}
\parskip=6pt % Espacement vertical entre les paragraphes

\begin{document}
\pagenumbering{gobble}  % Pas de numérotation
\begin{titlepage}
    \vspace*{50px}
    \includegraphics[height=80px]{Images/logo_phelma.pdf}
    \vspace*{-80px}
\begin{flushright}
%     \vspace*{60px}
    \includegraphics[height=65px]{Images/CIME.jpg}
\end{flushright}

\vspace*{2cm}

\begin{center}
\rule{\linewidth}{0.5mm}\\[0.4cm]
{\huge{\bfseries Compte Rendu}\\[0.4cm]
\textsc{TP Simulation électronique}\\[0.4cm]}
\rule{\linewidth}{0.5mm}\\[0.5cm]

\LARGE{\textsc{Nicolas Paillet, Félix Piédallu \& Giulia Rizzo}}\\[0.7cm]
\large{\textsc{2015-2016}}\\[2cm]

\Large{~}\\[1cm]
% \includegraphics[width=0.4\textwidth]{Images/CIME.jpg}\\[1cm]
%
% \large{}\\[2cm]
%

\end{center}
\end{titlepage}

\tableofcontents        % Table des matières avec liens, générée automatiquement.
\newpage
\pagenumbering{arabic}  % Numérotation de retour !

\chapter*{Introduction} \addcontentsline{toc}{chapter}{Introduction}
Nous avons vu lors de notre cours d'Optique Non Linéaire que certains cristaux ne se comportent pas de manière isotrope et linéaire lorsqu'on leur impose un rayonnement Laser.

Le but de ce TP est donc d'étudier le comportement d'un cristal de RbTiOPO$_4$ (RTP).

Dans une première partie nous avons utilisé un laser He:Ne de longueur d'onde $\lambda = 632.8nm$ pour faire apparaître le caractère anisotrope du cristal, ainsi que la séparation des deux composantes de polarisation de la lumière par le cristal (Chapitres \ref{OCLTheorie} et \ref{OCLExp}).

La seconde partie (Chapitre \ref{ONL}) portera sur l'utilisation d'un laser NdYAG ($\lambda = 1064nm$, invisible) qui nous a permis d'observer le caractère non linéaire du cristal par un doublage de fréquence de l'onde incidente vers la lumière visible.

% On pourrait fusionner les parties théorie et expérience, non ? Et si tu as d'autres idées de titres je suis preneur
\chapter{Optique Cristalline Linéaire (Théorie)} \label{OCLTheorie}
\section{Surface des indices}
\section{Aparallélisme des champs}
\section{Séparation des polarisations avec un cristal parallélépipédique}
\section{Séparation des polarisations avec un cristal cylindrique}

\chapter{Optique Cristalline Linéaire (Expérience)} \label{OCLExp}
\section{Mise en place de l'expérience}
\section{Orientation du repère optique}
\section{Mesure de la distance entre les spots de polarisations}

% Idem
\chapter{Optique Non-Linéaire: Conversion de fréquences} \label{ONL}
\section{Mise en place de l'expérience pour une SHG de type II}
\section{Mesure des angles d'accord de phase}

\chapter*{Conclusion} \addcontentsline{toc}{chapter}{Conclusion}

\end{document}
