\documentclass[a4paper,11pt]{report}
\usepackage[T1]{fontenc}
\usepackage[utf8]{inputenc}
\usepackage{lmodern}
\usepackage[francais]{babel}

\usepackage[usenames,dvipsnames,svgnames,table]{xcolor}
\usepackage[colorlinks,linkcolor={blue!30!black},citecolor={blue!50!black},urlcolor={blue!80!black}]{hyperref}
\usepackage{amsmath,array,graphicx,caption,lmodern,subcaption,tikz,url,xspace,wrapfig}
\usepackage{textcomp,rotating,epic,eepic,pdfpages}
\usepackage[top=2cm,left=2.5cm,right=2.5cm,bottom=2cm]{geometry} % Géométrie de la page, modifier selon le besoin
\usepackage[babel=true,kerning=true]{microtype}

\begin{document}

\pagenumbering{gobble}  % Pas de numérotation
\begin{titlepage}
    \vspace*{50px}
    \includegraphics[height=80px]{Images/logo_phelma.pdf}
    \vspace*{-80px}
\begin{flushright}
%     \vspace*{60px}
    \includegraphics[height=65px]{Images/CIME.jpg}
\end{flushright}

\vspace*{2cm}

\begin{center}
\rule{\linewidth}{0.5mm}\\[0.4cm]
{\huge{\bfseries Compte Rendu}\\[0.4cm]
\textsc{TP Simulation électronique}\\[0.4cm]}
\rule{\linewidth}{0.5mm}\\[0.5cm]

\LARGE{\textsc{Nicolas Paillet, Félix Piédallu \& Giulia Rizzo}}\\[0.7cm]
\large{\textsc{2015-2016}}\\[2cm]

\Large{~}\\[1cm]
% \includegraphics[width=0.4\textwidth]{Images/CIME.jpg}\\[1cm]
%
% \large{}\\[2cm]
%

\end{center}
\end{titlepage}

\tableofcontents        % Table des matières avec liens, générée automatiquement.
\newpage
\pagenumbering{arabic}  % Numérotation de retour !


\chapter*{Introduction}
\addcontentsline{toc}{chapter}{Introduction}

\section*{Paramètres utilisés}

\chapter{Rappels des équations}

\chapter{Mesure de la tension de seuil}
La tension de seuil est un paramètre important du transistor. %+blabla
\section{Méthode de la transconductance}

Cette méthode repose sur la caractéristique $I_d$ en fonction de $V_g$. En effet, ce tracé permet de déterminer assez facilement $V_T$. Cependant, une méthode uniquement visuelle n'est pas systématique et fortement dépendante de l'utilisateur. On utilise ainsi une méthode plus adaptée à la caractérisation. On trace la dérivée de $I_d(V_g)$, on détermine son maximum puis, on utilise la tangente à la courbe en ce maximum. Le croisement de cette tangente avec 0 nous donne $V_T$.
%schéma graphe

\subsection{Transistor Bulk}
%graphe
\subsection{Transistor FDSOI}
%graphe
\noindent On obtient ainsi le gain du transistor \[\beta=G_m=341\mu S\] Ce maximum est atteint pour $V_G=670mV$.
%graphe tangente

Le tracé de la tangente nous donne ainsi $V_T=450mV$.

\section{Méthode de la fonction Y}
On peut également utiliser une méthode différente pour déterminer $V_T$, à l'aide d'une fonction Y, définie par :
\[Y(V_g)=\dfrac{I_d(V_g)}{\sqrt{g_m(V_g)}}\]

Cette fonction est linéaire après le seuil, on peut alors approximer asymptotiquement plus précisément. On obtient alors $V_T$.
\subsection{Transistor Bulk}
%graphes
\subsection{Transistor FDSOI}
%graphes
%résultats
\section{Méthode du courant constant}
A $V_d$ fort il n'est plus possible de trouver grâce à la méthode de la transconductance. On utilise alors une méthode dite du courant constant. Elle consiste à trouver un courant de seuil $I_{d_{TH}}$ pour $V_d$ faible, qui correspond au courant à $V_T$, puis de considérer la relation : \[I_{d_{TH}}=I_{d_{DN}}\cdot\dfrac{W}{L}\]

$I_{d_{DN}}$ est un paramètre arbitraire, réutilisable qui permet de calculer $V_T$ pour plusieurs transistors.


\subsection{Méthode de la transconductance}


\subsection{Méthode de la fonction Y}

\chapter{Mesure du DIBL}
Le DIBL est également important à connaître %+blabla
\section{Transistor Bulk}

\section{Transistor FDSOI}

\chapter{Comparaison des architectures}

\chapter*{Conclusion}
\addcontentsline{toc}{chapter}{Conclusion}

Ça marche. %Mais il y a des cons promis.
\end{document}
