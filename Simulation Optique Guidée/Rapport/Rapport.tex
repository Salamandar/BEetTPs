\documentclass[a4paper,11pt]{report}
\usepackage[T1]{fontenc}
\usepackage[utf8]{inputenc}
\usepackage{lmodern}
\usepackage[francais]{babel}
\usepackage[usenames,dvipsnames,svgnames,table]{xcolor}
\usepackage[colorlinks,linkcolor={blue!30!black},citecolor={blue!50!black},urlcolor={blue!80!black}]{hyperref}
\usepackage{amsmath,array,graphicx,caption,lmodern,subcaption,tikz,url,xspace,wrapfig}
\usepackage{textcomp,rotating,epic,eepic,pdfpages}
\usepackage[top=2cm,left=2.5cm,right=2.5cm,bottom=2cm]{geometry} % Géométrie de la page, modifier selon le besoin
\usepackage[babel=true,kerning=true]{microtype}
\parskip=6pt % adds vertical space between paragraphs

\begin{document}

\pagenumbering{gobble}  % Pas de numérotation
\begin{titlepage}
    \vspace*{50px}
    \includegraphics[height=80px]{Images/logo_phelma.pdf}
    \vspace*{-80px}
\begin{flushright}
%     \vspace*{60px}
    \includegraphics[height=65px]{Images/CIME.jpg}
\end{flushright}

\vspace*{2cm}

\begin{center}
\rule{\linewidth}{0.5mm}\\[0.4cm]
{\huge{\bfseries Compte Rendu}\\[0.4cm]
\textsc{TP Simulation électronique}\\[0.4cm]}
\rule{\linewidth}{0.5mm}\\[0.5cm]

\LARGE{\textsc{Nicolas Paillet, Félix Piédallu \& Giulia Rizzo}}\\[0.7cm]
\large{\textsc{2015-2016}}\\[2cm]

\Large{~}\\[1cm]
% \includegraphics[width=0.4\textwidth]{Images/CIME.jpg}\\[1cm]
%
% \large{}\\[2cm]
%

\end{center}
\end{titlepage}

\tableofcontents        % Table des matières avec liens, générée automatiquement.
\newpage
\pagenumbering{arabic}  % Numérotation de retour !


\chapter*{Introduction}
\addcontentsline{toc}{chapter}{Introduction}

L'essor des télécommunications optiques n'est désormais plus à prouver. Omniprésentes dans la plupart des réseaux de télécommunication, les fibres optiques sont devenues incontournables. Elles sont d'autant plus intéressantes qu'il est possible de véhiculer plusieurs signaux à plusieurs longueurs d'ondes dans une seule et même fibre. \newline
Il est donc nécessaire de pouvoir séparer en fin de fibre les différents signaux, d'où l'utilisation de démultiplexeurs.

Le but de ce bureau d'étude est de simuler le fonctionnement d'un coupleur directionnel et de mesurer ses performances en démultiplexage de deux longueur d'ondes.

Nous mettrons en évidence les différents paramètres que l'on peut adapter pour modifier le fonctionnement du composant, tels que les indices effectifs $n_eff$, la distance entre les deux guides et la longueur du coupleur.
\newline
\indent Nous mesurerons enfin les performances du démultiplexeur, illustrées par l'isolation des deux bras aux longueurs d'ondes nominales de fonctionnement, ainsi que la tolérance en longueur d'onde du fonctionnement du dispositif et les effets de la courbure en entrée et sortie du coupleur.


\chapter{Étude théorique}

\section{Position du problème}
%schéma

\section{Calculs}%chiants
%calculs
\chapter{Dimensionnement}
\section{Intégration des paramètres d'utilisation}
% injection des paramètres dans les calculs
\section{Optimisation des paramètres}
% Determination des paramètres en fonction de la partie d'avant
\chapter{Simulation}
\section{Résultats de modélisation}
%Jolies images en 3D
\section{Validation des paramètres}
%Est-ce que c'est swag ou pas ?
\section{Calculs de puissance}
%Isolation et Pertes
\chapter*{Conclusion}
\addcontentsline{toc}{chapter}{Conclusion}

\appendix
\chapter{Scripts}
\section{Guide}
\section{Itérations}

\bibliographystyle{unsrt}
\bibliography{Biblio}


\end{document}
